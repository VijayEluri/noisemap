% This file is part of lamake
%
% Copyright (C) 2010 Nikos Maris
%
% This program is free software; you can redistribute it and/or modify it under
% the terms of the GNU General Public Licence as published by the Free Software
% Foundation; either version 2 of the Licence, or (at your option) any later
% version.
%
% This program is distributed in the hope that it will be useful, but WITHOUT
% ANY WARRANTY; without even the implied warranty of MERCHANTABILITY or FITNESS
% FOR A PARTICULAR PURPOSE.  See the GNU General Public Licence for more 
% details.
%
% You should have received a copy of the GNU General Public Licence along with
% this program; if not, write to the Free Software Foundation, Inc., 51 Franklin
% Street, Fifth Floor, Boston, MA  02110-1301, USA
%%%%%%%%%%%%%%%%%%%%%%%%%%%%%%%%%%%%%%%%%%%%%%%%%%%%
% This header file adds conditional compilation support to tex so you can process your tex file depending on the configuration on top of it.
% Sample usage of this file:
% % This file is part of lamake
%
% Copyright (C) 2010 Nikos Maris
%
% This program is free software; you can redistribute it and/or modify it under
% the terms of the GNU General Public Licence as published by the Free Software
% Foundation; either version 2 of the Licence, or (at your option) any later
% version.
%
% This program is distributed in the hope that it will be useful, but WITHOUT
% ANY WARRANTY; without even the implied warranty of MERCHANTABILITY or FITNESS
% FOR A PARTICULAR PURPOSE.  See the GNU General Public Licence for more 
% details.
%
% You should have received a copy of the GNU General Public Licence along with
% this program; if not, write to the Free Software Foundation, Inc., 51 Franklin
% Street, Fifth Floor, Boston, MA  02110-1301, USA
%%%%%%%%%%%%%%%%%%%%%%%%%%%%%%%%%%%%%%%%%%%%%%%%%%%%
% This header file adds conditional compilation support to tex so you can process your tex file depending on the configuration on top of it.
% Sample usage of this file:
% % This file is part of lamake
%
% Copyright (C) 2010 Nikos Maris
%
% This program is free software; you can redistribute it and/or modify it under
% the terms of the GNU General Public Licence as published by the Free Software
% Foundation; either version 2 of the Licence, or (at your option) any later
% version.
%
% This program is distributed in the hope that it will be useful, but WITHOUT
% ANY WARRANTY; without even the implied warranty of MERCHANTABILITY or FITNESS
% FOR A PARTICULAR PURPOSE.  See the GNU General Public Licence for more 
% details.
%
% You should have received a copy of the GNU General Public Licence along with
% this program; if not, write to the Free Software Foundation, Inc., 51 Franklin
% Street, Fifth Floor, Boston, MA  02110-1301, USA
%%%%%%%%%%%%%%%%%%%%%%%%%%%%%%%%%%%%%%%%%%%%%%%%%%%%
% This header file adds conditional compilation support to tex so you can process your tex file depending on the configuration on top of it.
% Sample usage of this file:
% % This file is part of lamake
%
% Copyright (C) 2010 Nikos Maris
%
% This program is free software; you can redistribute it and/or modify it under
% the terms of the GNU General Public Licence as published by the Free Software
% Foundation; either version 2 of the Licence, or (at your option) any later
% version.
%
% This program is distributed in the hope that it will be useful, but WITHOUT
% ANY WARRANTY; without even the implied warranty of MERCHANTABILITY or FITNESS
% FOR A PARTICULAR PURPOSE.  See the GNU General Public Licence for more 
% details.
%
% You should have received a copy of the GNU General Public Licence along with
% this program; if not, write to the Free Software Foundation, Inc., 51 Franklin
% Street, Fifth Floor, Boston, MA  02110-1301, USA
%%%%%%%%%%%%%%%%%%%%%%%%%%%%%%%%%%%%%%%%%%%%%%%%%%%%
% This header file adds conditional compilation support to tex so you can process your tex file depending on the configuration on top of it.
% Sample usage of this file:
% \input{controlLatexFlow}
% \enable{greek}																			% Commands with no definition can be used as C flags.
% %\newcommand{\recipient}{Αγαπητή κα.X,}		% commented so it is not used
% \opening{\ifthenels{recipient}{\recipient}{Αγαπητή κυρία/κύριε,}}

\usepackage{etoolbox}
\newrobustcmd*\enable[1]{ \global\csdef{#1}{} }			% \enable{greek} is like \newcommand{\greek}{} but can be used on input files of current scope
\newrobustcmd*\disable[1]{ \global\csundef{#1}{} }	% e.g. \csdef takes x and forms \x
% If you want empty lines in the 2nd or 3rd argument, put a comment so they are not empty.
\newrobustcmd*\ifthenels[3]{ \ifcsdef{#1}{#2}{#3} }						% #1 is a command without the backslash
\newrobustcmd*\ifthen[2]{ \ifthenels{#1}{#2}{} }							% false if command is not defined
\newrobustcmd*\ifthenelsempty[3]{ \ifstrempty{#1}{#2}{#3} }		% #1 is a string or a command

%\usepackage{ifthen}
%\newcommand{\turnon}[1]{\newcounter{#1}\setcounter{#1}{1}}
%\newcommand{\turnoff}[1]{\newcounter{#1}\setcounter{#1}{0}}
%\newcommand{\ifthen}[2]{\ifnum\value{#1}=1#2\fi} % if true, run given commands. example: \ifthen{academic}{HELLO}
%\newcommand{\ifthenels}[3]{\ifnum\value{#1}=1#2\else#3\fi}

% \enable{greek}																			% Commands with no definition can be used as C flags.
% %\newcommand{\recipient}{Αγαπητή κα.X,}		% commented so it is not used
% \opening{\ifthenels{recipient}{\recipient}{Αγαπητή κυρία/κύριε,}}

\usepackage{etoolbox}
\newrobustcmd*\enable[1]{ \global\csdef{#1}{} }			% \enable{greek} is like \newcommand{\greek}{} but can be used on input files of current scope
\newrobustcmd*\disable[1]{ \global\csundef{#1}{} }	% e.g. \csdef takes x and forms \x
% If you want empty lines in the 2nd or 3rd argument, put a comment so they are not empty.
\newrobustcmd*\ifthenels[3]{ \ifcsdef{#1}{#2}{#3} }						% #1 is a command without the backslash
\newrobustcmd*\ifthen[2]{ \ifthenels{#1}{#2}{} }							% false if command is not defined
\newrobustcmd*\ifthenelsempty[3]{ \ifstrempty{#1}{#2}{#3} }		% #1 is a string or a command

%\usepackage{ifthen}
%\newcommand{\turnon}[1]{\newcounter{#1}\setcounter{#1}{1}}
%\newcommand{\turnoff}[1]{\newcounter{#1}\setcounter{#1}{0}}
%\newcommand{\ifthen}[2]{\ifnum\value{#1}=1#2\fi} % if true, run given commands. example: \ifthen{academic}{HELLO}
%\newcommand{\ifthenels}[3]{\ifnum\value{#1}=1#2\else#3\fi}

% \enable{greek}																			% Commands with no definition can be used as C flags.
% %\newcommand{\recipient}{Αγαπητή κα.X,}		% commented so it is not used
% \opening{\ifthenels{recipient}{\recipient}{Αγαπητή κυρία/κύριε,}}

\usepackage{etoolbox}
\newrobustcmd*\enable[1]{ \global\csdef{#1}{} }			% \enable{greek} is like \newcommand{\greek}{} but can be used on input files of current scope
\newrobustcmd*\disable[1]{ \global\csundef{#1}{} }	% e.g. \csdef takes x and forms \x
% If you want empty lines in the 2nd or 3rd argument, put a comment so they are not empty.
\newrobustcmd*\ifthenels[3]{ \ifcsdef{#1}{#2}{#3} }						% #1 is a command without the backslash
\newrobustcmd*\ifthen[2]{ \ifthenels{#1}{#2}{} }							% false if command is not defined
\newrobustcmd*\ifthenelsempty[3]{ \ifstrempty{#1}{#2}{#3} }		% #1 is a string or a command

%\usepackage{ifthen}
%\newcommand{\turnon}[1]{\newcounter{#1}\setcounter{#1}{1}}
%\newcommand{\turnoff}[1]{\newcounter{#1}\setcounter{#1}{0}}
%\newcommand{\ifthen}[2]{\ifnum\value{#1}=1#2\fi} % if true, run given commands. example: \ifthen{academic}{HELLO}
%\newcommand{\ifthenels}[3]{\ifnum\value{#1}=1#2\else#3\fi}

% \enable{greek}																			% Commands with no definition can be used as C flags.
% %\newcommand{\recipient}{Αγαπητή κα.X,}		% commented so it is not used
% \opening{\ifthenels{recipient}{\recipient}{Αγαπητή κυρία/κύριε,}}

\usepackage{etoolbox}
\newrobustcmd*\enable[1]{ \global\csdef{#1}{} }			% \enable{greek} is like \newcommand{\greek}{} but can be used on input files of current scope
\newrobustcmd*\disable[1]{ \global\csundef{#1}{} }	% e.g. \csdef takes x and forms \x
% If you want empty lines in the 2nd or 3rd argument, put a comment so they are not empty.
\newrobustcmd*\ifthenels[3]{ \ifcsdef{#1}{#2}{#3} }						% #1 is a command without the backslash
\newrobustcmd*\ifthen[2]{ \ifthenels{#1}{#2}{} }							% false if command is not defined
\newrobustcmd*\ifthenelsempty[3]{ \ifstrempty{#1}{#2}{#3} }		% #1 is a string or a command

%\usepackage{ifthen}
%\newcommand{\turnon}[1]{\newcounter{#1}\setcounter{#1}{1}}
%\newcommand{\turnoff}[1]{\newcounter{#1}\setcounter{#1}{0}}
%\newcommand{\ifthen}[2]{\ifnum\value{#1}=1#2\fi} % if true, run given commands. example: \ifthen{academic}{HELLO}
%\newcommand{\ifthenels}[3]{\ifnum\value{#1}=1#2\else#3\fi}
